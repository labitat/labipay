\documentclass[a4paper,11pt]{article}
\usepackage[T1]{fontenc}

\title{Labipay Terminal API Documentation}
\usepackage[utf8]{inputenc}


\author{Jørgen Elgaard Larsen}


\begin{document}
\maketitle

\section{Terminal Communication}
There can be several types of terminals: Touch screen based, bar code based,
vending machines, etc. Terminals can be very dumb, or perform more complex tasks. 

All terminals communicates with the server using JSON over HTTP(S). 

Communication is initiated by the terminal making a request to the
server and ends with the server sending a response back to the terminal. Each
request/response is considered an atomic transaction.

If a terminal successfully sends a request but do not receive a response, it
should assume neither that the response is simply lost and the transaction
went through; nor that the request was lost and should be
retransmitted. In such a situation, the proper action depends on the semantics
of the transaction.

For example, if the transaction was just a user authorisation, the terminal
can simply try again. On the other hand, if it was a purchase transaction, the
terminal could try to ask for the account balance and see whether it has
changed since before the attempted transaction.

In a future version of the API, it may be possible for the terminal to mark
each transaction with an identifier and later ask whether that transaction
succeeded. But that is not implemented in this version of the API.


\section{Request And Response Format}

As mentioned above, requests must be made using JSON. The proper way is to use
an HTTP POST with the JSON request as the body.

All requests must contain the following fields:

\begin{table}[!h]
  \begin{tabular}{|l|l|}
    \hline
    request & The request type, e.g. ``auth'' \\\hline
    terminal & The terminal ID \\\hline
    seed & A random text, between 10 and 100 characters \\\hline
    checksum & A SHA-1 checksum of selected fields, the seed, and a shared secret \\\hline

  \end{tabular} 
  \caption{Common mandatory request fields}
\end{table}



\end{document}
