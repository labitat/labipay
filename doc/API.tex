\documentclass[a4paper,11pt]{article}
\usepackage[T1]{fontenc}

\title{Labipay Terminal API Documentation}
\usepackage[utf8]{inputenc}


\author{Jørgen Elgaard Larsen}

\newcommand{\code}[1]
   {\textsf{\mbox{#1}}}

\newcommand{\rightcellwidth}{25em}

\newcommand{\reqsection}[1]
    {\subsubsection{The \code{#1} Request}}

\newcommand{\partsection}[1]
           {\paragraph{#1}}

\begin{document}
\maketitle

\section{Terminal Communication}
There can be several types of terminals: Touch screen based, bar code based,
vending machines, etc. Terminals can be very dumb, or perform more complex tasks. 

All terminals communicates with the server using JSON over HTTP(S). 

Communication is initiated by the terminal making a request to the
server and ends with the server sending a response back to the terminal. Each
request/response is considered an atomic transaction.

If a terminal successfully sends a request but do not receive a response, it
should assume neither that the response is simply lost and the transaction
went through; nor that the request was lost and should be
retransmitted. In such a situation, the proper action depends on the semantics
of the transaction.

For example, if the transaction was just a user authorisation, the terminal
can simply try again. On the other hand, if it was a purchase transaction, the
terminal could try to ask for the account balance and see whether it has
changed since before the attempted transaction. Or, even better, use
the \code{lastTrans} request.


\section{Request And Response Format}\label{sec:rrFormat}

As mentioned above, requests must be made using JSON. The proper way is to use
an HTTP POST with the JSON request as the body.

All requests must contain at least the following fields:

\begin{table}[!ht]
  \begin{tabular}{|l|p{\rightcellwidth}|}
    \hline
    request & The request type, e.g. ``auth'' \\\hline
    terminal & The terminal ID \\\hline
    requestID & A positive integer identifying the request. Must be unique
      within the terminal -- i.e. a terminal must not send the same ID
      twice, but is not influenced by what other terminals have sent. If
      the terminal lacks persistent memory, it should only use
      increasing request ID's and use the \code{lastTrans} request
      when booting.\\\hline 
    session & A Session key obtained through an \code{auth} request or
      similar. Must be included in every request except when otherwise
      noted in the request specification.\\\hline
    renew\_session & If set to the value ``yes'', the session is
      renewed if possible.\\\hline
    checksum & A SHA-1 checksum of a concatenation of selected fields,
      the request id, and a shared secret \\\hline 
    version & The version of this protocol. For now, the only accepted
      value is ``1.0''. \\\hline 
  \end{tabular} 
  \caption{Common mandatory request fields}
\end{table}

Depending on the type of request, further fields are also needed. See
section \ref{sec:requestDetails} below. 


When the response has been received by the server, it is processed,
and a response is sent back. The response consists of a HTTP response
header, and a JSON representation of the response.

The response always contains at least these fields:

\begin{table}[!ht]
  \begin{tabular}{|l|p{\rightcellwidth}|}
    \hline
    status & A three-digit status code. If the transaction went
       through, the first digit is always 1. If the transaction did not
       go through, the first digit is never 1.
       For more details, see section \ref{sec:statusCodes}.\\\hline
    message & A status message, suitable for displying to humans.\\\hline
    shortMessage & Same as \code{message}, but guaranteed to be at
       most 16 characters long.\\\hline
    checksum & A SHA-1 checksum of the concatenation of status,
      selected fields and a shared secret \\\hline
    session & A session key to be included in future requests in the
      same session.\\\hline
    timeout & An integer representing the number of seconds until the
      current session expires.\\\hline
    version & The version of this protocol. For now, the only accepted
      value is ``1.0''. \\\hline 
  \end{tabular} 
  \caption{Common mandatory response fields}
\end{table}

Additional fields are used, depending on the request.

\subsection{Number formats}
Numbers must not contain thousand separators. For integer values, no
decimal point is allowed. For floating point numbers, use a dot (.) as
decimal point. For negative numbers, prepend with a dash (-) as a
minus sign. Do \emph{not} prepend positive numbers with a plus. 

Dates are formatted YYYY-MM-DD, time as HH:MM:SS (24h time). Where
nothing else is stated, time is UTC.


\section{Status Codes}\label{sec:statusCodes} 

As mentioned in section \ref{sec:rrFormat}, the response always
contains a three-digit status code to indicate whether the transaction went
through. The first digit of the status code indicates the status category:   

\begin{table}[!ht]
  \begin{tabular}{|l|p{\rightcellwidth}|}
    \hline
      1\emph{xx} & The transaction went through. \\\hline
      2\emph{xx} & User authorisation errors.\\\hline
      3\emph{xx} & Terminal authorisation errors.\\\hline
      4\emph{xx} & Request specific errors. See documentation for the
                   individual request type.\\\hline
      5\emph{xx} & Request errors. It will not help to re-send the
                   request.\\\hline
      6\emph{xx} & Server errors. The request may be valid, but the
                   server cannot handle it at the moment.\\\hline 
  \end{tabular} 
  \caption{Status codes categories}
\end{table}



Table \ref{tab:statusCodes} shows the status codes that are relevant
for all requests. 

\begin{table}[!hp]
  \begin{tabular}{|l|p{\rightcellwidth}|}
    \hline
      100  & OK \\\hline\hline
      200  & Generic authorisation error.\\\hline
      201  & User not authorised for this request type.\\\hline
      202  & User not authorised for this terminal.\\\hline
      203  & User unknown.\\\hline
      204  & Bad credentials.\\\hline
      205  & Session timed out.\\\hline
      300  & Unknown terminal.\\\hline
      301  & Terminal not authorised for this request
             type.\\\hline
      302  & Fraud detector triggered.\\\hline\hline
      500  & Generic request error.\\\hline
      501  & Malformed request.\\\hline
      502  & Unknown request type.\\\hline
      503  & Checksum error.\\\hline
      504  & Protocol version not supported.\\\hline
      505  & Missing fields in request.\\\hline
      506  & Duplicate request ID.\\\hline
      507  & Request ID too large.\\\hline
      508  & Request ID not a positive integer\\\hline
      509  & Field or fields has invalid content\\\hline\hline
      600  & Generic server error.\\\hline
      601  & Server busy - try again later.\\\hline
      602  & Could not connect to database\\\hline
      603  & Database error while handling the request.\\\hline
      604  & No network connection.\\\hline
      666  & Server on fire. Help!\\\hline
  \end{tabular} 
  \caption{Common response status codes}\label{tab:statusCodes}
\end{table}

Apart from these status codes, some request types has further possible
status codes. Please refer to the doocumentation for each individual
request type for details. 

\clearpage

\section{Request And Response Details}\label{sec:requestDetails} 


%%%%%%%%%%
\subsection{User Authorisation}

\reqsection{cardauth}

Authorise a user from his magnetic card.

\partsection{Request}
\begin{table}[!hb]
  \begin{tabular}{|l|p{\rightcellwidth}|}
    \hline
    cardhash & A MD5 hash of the card's information on track 2.\\\hline
    withPin  & Set to ``yes'' if the cardhash includes the user's PIN
               number. Optional, default ``no''. \\\hline
  \end{tabular} 
  \caption{Extra fields in \code{cardauth} request}
\end{table}

Checksum fields: request, terminal, requestID, cardhash, shared secret.

Does not require a session.

\partsection{Response}
\begin{table}[!ht]
  \begin{tabular}{|l|p{\rightcellwidth}|}
    \hline
      userid  & The user ID of the customer \\\hline
      handle  & The handle (login name) of the user.\\\hline
      name    & The real world name of the user, if known.\\\hline
      account & The account number for the card\\\hline
      accountName & The name of the account\\\hline
      balance & The balance for the account\\\hline
  \end{tabular} 
  \caption{Extra fields in \code{cardauth} response}
\end{table}

Checksum fields: status, request, terminal, requestID, account, shared secret.


\begin{table}[!ht]
  \begin{tabular}{|l|p{\rightcellwidth}|}
    \hline
      250  & PIN required\\\hline
      251  & PIN not required --- try without PIN\\\hline
      252  & Authorisation type not allowed. Use the \code{auth}
      request instead.\\\hline
  \end{tabular} 
  \caption{Status codes for the \code{cardauth} response}
\end{table}


%%%%%%%%%%

\reqsection{auth}

Authorise a user by username and password.

This is a \emph{very} sensitive transaction and must only be sent
over a trusted line, e.g. HTTPS.

\partsection{Request}
\begin{table}[!ht]
  \begin{tabular}{|l|p{\rightcellwidth}|}
    \hline
    usename & The user name in cleartext.\\\hline
    password & The user's password in cleartext.\\\hline
  \end{tabular} 
  \caption{Extra fields in \code{auth} request}
\end{table}

Checksum fields: request, terminal, requestID, username, password,
shared secret. 

Does not require a session.

\partsection{Response}
\begin{table}[!ht]
  \begin{tabular}{|l|p{\rightcellwidth}|}
    \hline
      userid  & The user ID of the customer \\\hline
      handle  & The handle (login name) of the user.\\\hline
      name    & The real world name of the user, if known.\\\hline
      accounts & An array of JSON objects each representing
                 an account. Attributes are: number, name, balance.\\\hline
       
  \end{tabular} 
  \caption{Extra fields in \code{auth} response}
\end{table}

Checksum fields: status, request, terminal, requestID, userid, shared secret.

No extra status codes.


%%%%%%%%%%

\reqsection{logout}

Terminate a session.

\partsection{Request}
\begin{table}[!ht]
  \begin{tabular}{|l|p{\rightcellwidth}|}
    \hline
    session & The session to terminate.\\\hline
  \end{tabular} 
  \caption{Extra fields in \code{logout} request}
\end{table}

Checksum fields: request, terminal, requestID, session,
shared secret. 


\partsection{Response}
No extra fields.

Checksum fields: status, request, terminal, requestID, shared secret.

No extra status codes.

Does not return a session key.

%%%%%%%%%%

\subsection{Purchase Transactions}

\reqsection{buy}

\partsection{Request}
\begin{table}[!ht]
  \begin{tabular}{|l|p{\rightcellwidth}|}
    \hline
     account & The account to charge. For cash sale, use account
               number 0.\\\hline
     product & The product number to buy.\\
     \multicolumn{2}{l}{\emph{- or -}}\\
     ean     & The EAN code of the product to buy.\\\hline
     quantum & The amount to buy of the product.\\\hline
     gamble  & If set to ``yes'', enable gamble mode. Default ``no''.\\\hline 
  \end{tabular} 
  \caption{Extra fields in \code{buy} request}
\end{table}

Checksum fields: request, terminal, requestID, account, product/ean,
quantum, shared secret.

\partsection{Response}
\begin{table}[!ht]
  \begin{tabular}{|l|p{\rightcellwidth}|}
    \hline
     price         & The price of the product, per unit\\\hline
     totalprice    & The total price of this purchase.\\\hline
     newBalance    & The new balance on the user's account.\\\hline
     productName   & Name of the product. \\\hline
     productShortName & Name of the product, at most 16 characters long.\\\hline
     productUnit   & The unit of the product, e.g. ``m''. \\\hline
     productNumber & The product number. \\\hline
  \end{tabular} 
  \caption{Extra fields in \code{buy} response}
\end{table}

Checksum fields: status, request, terminal, requestID, productshotNme,
newBalance, shared secret.

\begin{table}[!ht]
  \begin{tabular}{|l|p{\rightcellwidth}|}
    \hline
      451  & Sold out. No purchase possible.\\\hline
      452  & Gambling not allowed.\\\hline
      453  & No credit for the user.\\\hline
      454  & Cash sale not allowed for this product.\\\hline
      455  & Unknown product. \\\hline
      456  & Only integer quantum allowed. \\\hline
      457  & Quantum too large
  \end{tabular} 
  \caption{Status codes for the \code{buy} response}
\end{table}

%%%%%%%%%%

\reqsection{list}

Retrieve information on products in the database. Only products
relevant for this terminal is retrieved.

\partsection{Request}
\begin{table}[!ht]
  \begin{tabular}{|l|p{\rightcellwidth}|}
    \hline
     categories & An array of category ID's. Only show products in
                  these categories. Optional, default show all top level
                  categories.\\\hline 
     shops      & An array of shop ID's. Only show products for these
                  shops. Optional, default no constraint.  \\\hline
     levels     & Only retrieve data for $n$ levels. Optional, default all
                  levels. If set to 0, no product info is returned.\\\hline 
     shopInfo   & What shops to include information about. If set to
                  ``none'', no shop informtion is included. If set to
                  ``all'', information about all shops is included. If
                  set to ``relevant'', only the shops for which there
                  are products returned, are included. Optional,
                  default ``all''.  
  \end{tabular} 
  \caption{Extra fields in \code{list} request}
\end{table}

Please note that setting none of the above fields will result in a
full dump of the database for this terminal. Do not do that lightly,
as it may potentially be a large list. 

Checksum fields: request, terminal, requestID, shared secret.

\partsection{Response}
\begin{table}[!ht]
  \begin{tabular}{|l|p{\rightcellwidth}|}
    \hline
      shops & An array of shop objects. Each object has \code{shopID} and
             \code{shopName}\\\hline
      categories & An array of categories. Each category has a
                   \code{categoryId}, \code{categoryName} and \code{products}.
                   It also has \code{categories}, which is an array of sub
                   categories.

                   The \code{products} is an array of products, each containing
                   \code{productID}, \code{productName},
                   \code{productShortName}, \code{productUnit},
                   \code{productPrice}, \code{integerQuantum} (yes/no),
                   \code{productEAN}, \code{shopID}, \code{categoryID}. \\\hline
  \end{tabular} 
  \caption{Extra fields in \code{list} response}
\end{table}

Checksum fields: request, terminal, requestID, shared secret.

\begin{table}[!ht]
  \begin{tabular}{|l|p{\rightcellwidth}|}
    \hline
      353  & Listing not allowed for this terminal \\\hline
      460  & Unknown shop in list.\\\hline
      461  & Unknown category in list\\\hline
      462  & Too many levels\\\hline
      463  & Illegal value in field\\\hline
  \end{tabular} 
  \caption{Status codes for the \code{list} response}
\end{table}

%%%%%%%%%%

\reqsection{informSales}

\partsection{Request}
\begin{table}[!ht]
  \begin{tabular}{|l|p{\rightcellwidth}|}
    \hline
       \\\hline
  \end{tabular} 
  \caption{Extra fields in \code{} request}
\end{table}

Checksum fields: request, terminal, requestID, , shared secret.

\partsection{Response}
\begin{table}[!ht]
  \begin{tabular}{|l|p{\rightcellwidth}|}
    \hline
       \\\hline
  \end{tabular} 
  \caption{Extra fields in \code{} response}
\end{table}

Checksum fields: request, terminal, requestID, , shared secret.

\begin{table}[!ht]
  \begin{tabular}{|l|p{\rightcellwidth}|}
    \hline
      100  & OK \\\hline
      201  & \\\hline
      202  & \\\hline
      203  & \\\hline
  \end{tabular} 
  \caption{Status codes for the \code{} response}
\end{table}

%%%%%%%%%%

\reqsection{lastTrans}

\partsection{Request}
\begin{table}[!ht]
  \begin{tabular}{|l|p{\rightcellwidth}|}
    \hline
       \\\hline
  \end{tabular} 
  \caption{Extra fields in \code{} request}
\end{table}

Checksum fields: request, terminal, requestID, , shared secret.

\partsection{Response}
\begin{table}[!ht]
  \begin{tabular}{|l|p{\rightcellwidth}|}
    \hline
       \\\hline
  \end{tabular} 
  \caption{Extra fields in \code{} response}
\end{table}

Checksum fields: request, terminal, requestID, , shared secret.

%%%%%%%%%%

\reqsection{resetTrans}

\partsection{Request}
\begin{table}[!ht]
  \begin{tabular}{|l|p{\rightcellwidth}|}
    \hline
       \\\hline
  \end{tabular} 
  \caption{Extra fields in \code{} request}
\end{table}

Checksum fields: request, terminal, requestID, , shared secret.

\partsection{Response}
\begin{table}[!ht]
  \begin{tabular}{|l|p{\rightcellwidth}|}
    \hline
     newValue & The next request ID value to use.  \\\hline
  \end{tabular} 
  \caption{Extra fields in \code{} response}
\end{table}

Checksum fields: request, terminal, requestID, , shared secret.

%%%%%%%%%%

\reqsection{addMoney}

\partsection{Request}
\begin{table}[!ht]
  \begin{tabular}{|l|p{\rightcellwidth}|}
    \hline
       \\\hline
  \end{tabular} 
  \caption{Extra fields in \code{} request}
\end{table}

Checksum fields: request, terminal, requestID, , shared secret.

\partsection{Response}
\begin{table}[!ht]
  \begin{tabular}{|l|p{\rightcellwidth}|}
    \hline
       \\\hline
  \end{tabular} 
  \caption{Extra fields in \code{} response}
\end{table}

Checksum fields: request, terminal, requestID, , shared secret.

%%%%%%%%%%

\reqsection{transfer}

\partsection{Request}
\begin{table}[!ht]
  \begin{tabular}{|l|p{\rightcellwidth}|}
    \hline
       \\\hline
  \end{tabular} 
  \caption{Extra fields in \code{} request}
\end{table}

Checksum fields: request, terminal, requestID, , shared secret.

\partsection{Response}
\begin{table}[!ht]
  \begin{tabular}{|l|p{\rightcellwidth}|}
    \hline
       \\\hline
  \end{tabular} 
  \caption{Extra fields in \code{} response}
\end{table}

Checksum fields: request, terminal, requestID, , shared secret.

%%%%%%%%%%

\reqsection{listProducts}

\partsection{Request}
\begin{table}[!ht]
  \begin{tabular}{|l|p{\rightcellwidth}|}
    \hline
       \\\hline
  \end{tabular} 
  \caption{Extra fields in \code{} request}
\end{table}

Checksum fields: request, terminal, requestID, shared secret.

\partsection{Response}
\begin{table}[!ht]
  \begin{tabular}{|l|p{\rightcellwidth}|}
    \hline
       \\\hline
  \end{tabular} 
  \caption{Extra fields in \code{} response}
\end{table}

Checksum fields: request, terminal, requestID, , shared secret.

%%%%%%%%%%

\reqsection{registerCard}

\partsection{Request}
\begin{table}[!ht]
  \begin{tabular}{|l|p{\rightcellwidth}|}
    \hline
       \\\hline
  \end{tabular} 
  \caption{Extra fields in \code{} request}
\end{table}

Checksum fields: request, terminal, requestID, , shared secret.

\partsection{Response}
\begin{table}[!ht]
  \begin{tabular}{|l|p{\rightcellwidth}|}
    \hline
       \\\hline
  \end{tabular} 
  \caption{Extra fields in \code{} response}
\end{table}

Checksum fields: request, terminal, requestID, , shared secret.

%%%%%%%%%%

\reqsection{revertBuy}

\partsection{Request}
\begin{table}[!ht]
  \begin{tabular}{|l|p{\rightcellwidth}|}
    \hline
       \\\hline
  \end{tabular} 
  \caption{Extra fields in \code{} request}
\end{table}

Checksum fields: request, terminal, requestID, , shared secret.

\partsection{Response}
\begin{table}[!ht]
  \begin{tabular}{|l|p{\rightcellwidth}|}
    \hline
       \\\hline
  \end{tabular} 
  \caption{Extra fields in \code{} response}
\end{table}

Checksum fields: request, terminal, requestID, , shared secret.

%%%%%%%%%%

\reqsection{priceCheck}

\partsection{Request}
\begin{table}[!ht]
  \begin{tabular}{|l|p{\rightcellwidth}|}
    \hline
       \\\hline
  \end{tabular} 
  \caption{Extra fields in \code{} request}
\end{table}

Checksum fields: request, terminal, requestID, , shared secret.

\partsection{Response}
\begin{table}[!ht]
  \begin{tabular}{|l|p{\rightcellwidth}|}
    \hline
       \\\hline
  \end{tabular} 
  \caption{Extra fields in \code{} response}
\end{table}

Checksum fields: request, terminal, requestID, , shared secret.



%%%%%%%%%%

\reqsection{checkCertified}

\partsection{Request}
\begin{table}[!ht]
  \begin{tabular}{|l|p{\rightcellwidth}|}
    \hline
       \\\hline
  \end{tabular} 
  \caption{Extra fields in \code{} request}
\end{table}

Checksum fields: request, terminal, requestID, , shared secret.

\partsection{Response}
\begin{table}[!ht]
  \begin{tabular}{|l|p{\rightcellwidth}|}
    \hline
       \\\hline
  \end{tabular} 
  \caption{Extra fields in \code{} response}
\end{table}

Checksum fields: request, terminal, requestID, , shared secret.



%%%%%%%%%%

\reqsection{ping}

\partsection{Request}
\begin{table}[!ht]
  \begin{tabular}{|l|p{\rightcellwidth}|}
    \hline
       \\\hline
  \end{tabular} 
  \caption{Extra fields in \code{} request}
\end{table}

Checksum fields: request, terminal, requestID, , shared secret.

\partsection{Response}
\begin{table}[!ht]
  \begin{tabular}{|l|p{\rightcellwidth}|}
    \hline
       \\\hline
  \end{tabular} 
  \caption{Extra fields in \code{} response}
\end{table}

Checksum fields: request, terminal, requestID, , shared secret.



\section{Vulnrablities}



\end{document}
